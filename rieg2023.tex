% Options for packages loaded elsewhere
\PassOptionsToPackage{unicode}{hyperref}
\PassOptionsToPackage{hyphens}{url}
\PassOptionsToPackage{dvipsnames,svgnames,x11names}{xcolor}
%
\documentclass[
  11,
  a4paperpaper,
]{article}

\usepackage{amsmath,amssymb}
\usepackage{setspace}
\usepackage{iftex}
\ifPDFTeX
  \usepackage[T1]{fontenc}
  \usepackage[utf8]{inputenc}
  \usepackage{textcomp} % provide euro and other symbols
\else % if luatex or xetex
  \usepackage{unicode-math}
  \defaultfontfeatures{Scale=MatchLowercase}
  \defaultfontfeatures[\rmfamily]{Ligatures=TeX,Scale=1}
\fi
\usepackage{lmodern}
\ifPDFTeX\else  
    % xetex/luatex font selection
  \setmainfont[]{Arial}
\fi
% Use upquote if available, for straight quotes in verbatim environments
\IfFileExists{upquote.sty}{\usepackage{upquote}}{}
\IfFileExists{microtype.sty}{% use microtype if available
  \usepackage[]{microtype}
  \UseMicrotypeSet[protrusion]{basicmath} % disable protrusion for tt fonts
}{}
\makeatletter
\@ifundefined{KOMAClassName}{% if non-KOMA class
  \IfFileExists{parskip.sty}{%
    \usepackage{parskip}
  }{% else
    \setlength{\parindent}{0pt}
    \setlength{\parskip}{6pt plus 2pt minus 1pt}}
}{% if KOMA class
  \KOMAoptions{parskip=half}}
\makeatother
\usepackage{xcolor}
\usepackage[top=25mm,left=25mm,right=25mm,bottom=20mm,heightrounded]{geometry}
\setlength{\emergencystretch}{3em} % prevent overfull lines
\setcounter{secnumdepth}{5}
% Make \paragraph and \subparagraph free-standing
\ifx\paragraph\undefined\else
  \let\oldparagraph\paragraph
  \renewcommand{\paragraph}[1]{\oldparagraph{#1}\mbox{}}
\fi
\ifx\subparagraph\undefined\else
  \let\oldsubparagraph\subparagraph
  \renewcommand{\subparagraph}[1]{\oldsubparagraph{#1}\mbox{}}
\fi


\providecommand{\tightlist}{%
  \setlength{\itemsep}{0pt}\setlength{\parskip}{0pt}}\usepackage{longtable,booktabs,array}
\usepackage{calc} % for calculating minipage widths
% Correct order of tables after \paragraph or \subparagraph
\usepackage{etoolbox}
\makeatletter
\patchcmd\longtable{\par}{\if@noskipsec\mbox{}\fi\par}{}{}
\makeatother
% Allow footnotes in longtable head/foot
\IfFileExists{footnotehyper.sty}{\usepackage{footnotehyper}}{\usepackage{footnote}}
\makesavenoteenv{longtable}
\usepackage{graphicx}
\makeatletter
\def\maxwidth{\ifdim\Gin@nat@width>\linewidth\linewidth\else\Gin@nat@width\fi}
\def\maxheight{\ifdim\Gin@nat@height>\textheight\textheight\else\Gin@nat@height\fi}
\makeatother
% Scale images if necessary, so that they will not overflow the page
% margins by default, and it is still possible to overwrite the defaults
% using explicit options in \includegraphics[width, height, ...]{}
\setkeys{Gin}{width=\maxwidth,height=\maxheight,keepaspectratio}
% Set default figure placement to htbp
\makeatletter
\def\fps@figure{htbp}
\makeatother
% definitions for citeproc citations
\NewDocumentCommand\citeproctext{}{}
\NewDocumentCommand\citeproc{mm}{%
  \begingroup\def\citeproctext{#2}\cite{#1}\endgroup}
\makeatletter
 % allow citations to break across lines
 \let\@cite@ofmt\@firstofone
 % avoid brackets around text for \cite:
 \def\@biblabel#1{}
 \def\@cite#1#2{{#1\if@tempswa , #2\fi}}
\makeatother
\newlength{\cslhangindent}
\setlength{\cslhangindent}{1.5em}
\newlength{\csllabelwidth}
\setlength{\csllabelwidth}{3em}
\newenvironment{CSLReferences}[2] % #1 hanging-indent, #2 entry-spacing
 {\begin{list}{}{%
  \setlength{\itemindent}{0pt}
  \setlength{\leftmargin}{0pt}
  \setlength{\parsep}{0pt}
  % turn on hanging indent if param 1 is 1
  \ifodd #1
   \setlength{\leftmargin}{\cslhangindent}
   \setlength{\itemindent}{-1\cslhangindent}
  \fi
  % set entry spacing
  \setlength{\itemsep}{#2\baselineskip}}}
 {\end{list}}
\usepackage{calc}
\newcommand{\CSLBlock}[1]{\hfill\break#1\hfill\break}
\newcommand{\CSLLeftMargin}[1]{\parbox[t]{\csllabelwidth}{\strut#1\strut}}
\newcommand{\CSLRightInline}[1]{\parbox[t]{\linewidth - \csllabelwidth}{\strut#1\strut}}
\newcommand{\CSLIndent}[1]{\hspace{\cslhangindent}#1}

\makeatletter
\makeatother
\makeatletter
\makeatother
\makeatletter
\@ifpackageloaded{caption}{}{\usepackage{caption}}
\AtBeginDocument{%
\ifdefined\contentsname
  \renewcommand*\contentsname{Table of contents}
\else
  \newcommand\contentsname{Table of contents}
\fi
\ifdefined\listfigurename
  \renewcommand*\listfigurename{List of Figures}
\else
  \newcommand\listfigurename{List of Figures}
\fi
\ifdefined\listtablename
  \renewcommand*\listtablename{List of Tables}
\else
  \newcommand\listtablename{List of Tables}
\fi
\ifdefined\figurename
  \renewcommand*\figurename{Figure}
\else
  \newcommand\figurename{Figure}
\fi
\ifdefined\tablename
  \renewcommand*\tablename{Table}
\else
  \newcommand\tablename{Table}
\fi
}
\@ifpackageloaded{float}{}{\usepackage{float}}
\floatstyle{ruled}
\@ifundefined{c@chapter}{\newfloat{codelisting}{h}{lop}}{\newfloat{codelisting}{h}{lop}[chapter]}
\floatname{codelisting}{Listing}
\newcommand*\listoflistings{\listof{codelisting}{List of Listings}}
\makeatother
\makeatletter
\@ifpackageloaded{caption}{}{\usepackage{caption}}
\@ifpackageloaded{subcaption}{}{\usepackage{subcaption}}
\makeatother
\makeatletter
\makeatother
\ifLuaTeX
\usepackage[bidi=basic]{babel}
\else
\usepackage[bidi=default]{babel}
\fi
\babelprovide[main,import]{english}
\ifPDFTeX
\else
\babelfont{rm}[]{Arial}
\fi
% get rid of language-specific shorthands (see #6817):
\let\LanguageShortHands\languageshorthands
\def\languageshorthands#1{}
\ifLuaTeX
  \usepackage{selnolig}  % disable illegal ligatures
\fi
\IfFileExists{bookmark.sty}{\usepackage{bookmark}}{\usepackage{hyperref}}
\IfFileExists{xurl.sty}{\usepackage{xurl}}{} % add URL line breaks if available
\urlstyle{same} % disable monospaced font for URLs
\hypersetup{
  pdftitle={socioeconomic disruption by artificial intelligence},
  pdflang={en},
  colorlinks=true,
  linkcolor={blue},
  filecolor={Maroon},
  citecolor={Blue},
  urlcolor={Blue},
  pdfcreator={LaTeX via pandoc}}

\title{socioeconomic disruption by artificial intelligence}
\usepackage{etoolbox}
\makeatletter
\providecommand{\subtitle}[1]{% add subtitle to \maketitle
  \apptocmd{\@title}{\par {\large #1 \par}}{}{}
}
\makeatother
\subtitle{A comparative analysis between industries in the European
Union}
\author{}
\date{}

\begin{document}
\maketitle
\setstretch{1.5}
\pagenumbering{Roman}

\newpage{}

\tableofcontents

\newpage{}

\listoffigures

\newpage{}

\listoftables

\newpage{}

\pagenumbering{arabic}

\section{Abstract}\label{abstract}

\section{Introduction}\label{introduction}

Mokyr et al. (\citeproc{ref-mokyr_history_2015}{2015, p. 32}) identifies
two forms of technological anxiety, the fear of labor displacement
through technology and and the fear of morally negative applications
resulting in declining welfare. The majority of the US population has
been found to assess the potential impact of automation as unfavorable
rather than beneficial
(\citeproc{ref-anderson_automation_2017}{Anderson, 2017}).

Since AI is a still fairly new topic in the literature and has only seen
real increase in dominance and interest in recent years
(\citeproc{ref-acemoglu_ai_2020}{Acemoglu et al., 2020a, p. 23f}.), is
is worth noting the effects of previous technologies as the adoption of
machines (specifically often industrial robots
(\citeproc{ref-acemoglu_robots_2020}{Acemoglu and Restrepo, 2020a}; see
for example \citeproc{ref-graetz_robots_2015}{Graetz and Michaels,
2015})) and software (also referred to as
computerization(\citeproc{ref-autor_growth_2013}{Autor and Dorn, 2013};
\citeproc{ref-frey_future_2017}{Frey and Osborne, 2017}; see for example
\citeproc{ref-pajarinen_computerization_2015}{Pajarinen et al., 2015}))
have been seen as previous stages in the evolution of automation with AI
composing the next stage (\citeproc{ref-acemoglu_harms_2021}{Acemoglu,
2021, p. 19}). Furthermore, these technologies have been summarized
under the umbrella term ``automation''
(\citeproc{ref-mann_benign_2018}{Mann and Püttmann, 2018, p. 40})
indicating common characteristics and thereby - possibly - common
effects.

\subsection{Effects of Automation on
labor}\label{effects-of-automation-on-labor}

In a 2018 study, the introduction of automation technology was found to
have positive effects on employment gains, but only within the same
commuting zone (\citeproc{ref-mann_benign_2018}{Mann and Püttmann, 2018,
p. 26}). These findings contradict the results from Autor et al.
(\citeproc{ref-autor_untangling_2015}{2015}) {[}p.~632{]}, that found no
relation between exposure to automation and employment as a whole but
found a significant decline in employment related to routine tasks in
the non-manufacturing sector (p.~641). Graetz and Michaels
(\citeproc{ref-graetz_robots_2015}{2015, p. 766}) found no relationship
between the usage of industrial robots and net employment. However,
usage of industrial robots was found to lower employment of low-skilled
workers. However, a later study also looking at employment effects
induced by usage of industrial robots found a significant decline of
employment as well as a reduction in wages related to robot exposure
within a commuting zone (\citeproc{ref-acemoglu_robots_2020}{Acemoglu
and Restrepo, 2020a, pp. 2215f, 2218}). Dauth et al.
(\citeproc{ref-dauth_german_2017}{2017}) {[}p.~25{]} found no relation
between robot exposure and employment in the German market. A few years
later, Dauth et al. (\citeproc{ref-dauth_adjustment_2021}{2021, p.
3126ff}) found robot exposure to lead to within-firm and between-firm
job displacement, with displaced workers having difficulties
reallocating their jobs within the same industry, leading to a migration
of workers from manufacturing (where robot exposure is most present) to
the service sector. They also exhibited that a lack of worker
protections (for example unionization or tenure) is related to greater
displacement. These results were also confirmed by Boustan et al.
(\citeproc{ref-boustan_automation_2022}{2022}) {[}p.~21, 23{]} who
observed that displaced workers acquire new skills and concluded job
displacement by automation to be less discernible amongst unionized and
high-skilled workers. Similarily, Acemoglu and Restrepo
(\citeproc{ref-acemoglu_robots_2020}{2020a, p. 2215f}., 2218) provided
evidence showing automation (adoption of industrial robots) within a
commuting zone (local labor market) relating to significant declines in
employment as well as wages. By studying 53 developing countries, Cirera
and Sabetti (\citeproc{ref-cirera_effects_2019}{2019, p. 172}) did not
find a relationship between exposure to automation and firm level
employment. Hoewever, while a net effect on employment was absent, in
line with the aforementioned literature, they did find automation to
alter the composition of tasks and skills within firms (p.~172).

In a purely theoretical approach to the effects of automation on labor,
Acemoglu and Restrepo (\citeproc{ref-acemoglu_low-skill_2017}{2017})
{[}p.~12, 15{]} concluded that automation leads to labor displacement
and the displacement of low skilled-labor leading to an increase in the
wage gap (pay gap between low-skilled and high-skilled workers) while
the displacement of high-skilled labor is followed by a reduction in the
wage gap as high-skill labor reallocates into medium- and low-skilled
occupations. This reallocation from displaced high-skill labor into
lower skilled occupations has also been shown by Beaudry et al.
(\citeproc{ref-beaudry_great_2016}{2016, p. 21}) who studied the effects
on labor when prices for specific types of labor fall - as is induced
when substitution (through technology) becomes economically viable.
While labor displacement induced by the introduction of automation is
followed by increased inequality between low-skill and high-skilled
labor in the short run (\citeproc{ref-acemoglu_race_2018}{Acemoglu and
Restrepo, 2018a, p. 1519}), the creation of new tasks - that is followed
by increased productivity gains from automation - is seen to reduce this
gap in the long run (p.~1521). However, this positive outlook of a net
positive on employment only holds true as long as the productivity
effects which accompany the adoption of automation technologies offset
the displacement effects incurred in the first place - and should the
offset be insufficient, automation is found to negatively impact the
demand for labor and its wages
(\citeproc{ref-acemoglu_artificial_2018}{Acemoglu and Restrepo, 2018b,
p. 227}). There is also growing evidence suggesting automation to cause
a decline in real wages of low-skilled workers, for example Acemoglu and
Restrepo (\citeproc{ref-acemoglu_unpacking_2020}{2020b}) {[}p.~360f.{]}
found strong relationships between the adoption of automation technology
and wages. Acemoglu and Restrepo
(\citeproc{ref-acemoglu_tasks_2022}{2022, p. 1993}) found a relationship
between labor displacement and a decrease in relative wages, concluding
automation to cause an increase in wage inequality (p.~1998). Automation
is also attributed to the decline in the demand for labor in the US over
recent decades (\citeproc{ref-acemoglu_automation_2019}{Acemoglu and
Restrepo, 2019, p. 21}).

Furthermore, Arntz et al. (\citeproc{ref-arntz_risk_2016}{2016})
{[}p.~14f.{]} studying 21 OECD countries found 9\% in the US, and over
all countries studies a 6-12\% high risk of employment to be
substitutable for automation, while Acemoglu and Autor
(\citeproc{ref-acemoglu_chapter_2011}{2011, p. 61}) came to the
conclusion that labor displacement by machines mostly affects routine
tasks.

\subsection{Effects of
computerization}\label{effects-of-computerization}

In a study from Finland, Pajarinen et al.
(\citeproc{ref-pajarinen_computerization_2015}{2015}) found that
computerization is likely to place high risk of displacement on 35\% of
the Finish labor market {[}p.~5{]}, 33\% of Norwegian labor (p.~5) as
well as 49\% in the US {[}p.~5{]}. Frey and Osborne
(\citeproc{ref-frey_future_2017}{2017, p. 41}) found 47\% of US
employment to have a a high risk suitability for substitution by
computerization. They further classify the process of automation into
two ``waves'' with the first wave affecting routine tasks
(transportation, logistics, office, and administration) {[}p.~41{]}
followed by a second wave that, once technological obstacles are
overcome, will effect the jobs involving creative or abstract tasks
{[}p.~43{]}. Evidence also suggests computerization to significantly
induce labor displacement from occupations relying on routine tasks into
higher-skilled occupations as well as low-skilled service occupations
(\citeproc{ref-autor_growth_2013}{Autor and Dorn, 2013, p. 1573})

\subsection{Effects of AI}\label{effects-of-ai}

Brynjolfsson et al. (\citeproc{ref-brynjolfsson_what_2018}{2018})
{[}p.~46{]} found that machine learning affects different types of tasks
than earlier forms of automation. A year later, in a study comparing the
impact of AI on the job market between industries, Webb
(\citeproc{ref-webb_impact_2019}{2019, p. 46}) shows that AI affects
mostly the highly educated workforce and that this group is affected
significantly more by AI than the presence of software or robots. Under
the assumption that the current trend in technological evolution is set
to continue, the speed of labor displacement through technological
innovation is found likely to outpace the speed at which labor can be
relocated (\citeproc{ref-mokyr_history_2015}{Mokyr et al., 2015, p.
43f}.). By constructing impact scores of Artificial Intelligence on
occupations, Felten et al. (\citeproc{ref-felten_effect_2019}{2019})
found low-income occupations to experience a decline in wage growth that
is attributed to the increased presence of AI and middle and high-income
occupations to experience an increase in wage growth {[}p.~6{]}.
Furthermore, the authors found found that occupations with a medium and
high degree of automation (degree of automation being the presence of
automation technologies - not just AI) positively correlate with
employment when exposed to Artificial Intelligence, while they did not
find any relationship for occupations already exhibiting a low degree of
automation {[}p.~5{]}.

It has also been noted that the presence of Artificial Intelligence does
not have a linear impact on labor but depends on influencing factors,
such as price elasticity, complementarities, or elasticity of labor that
govern the implementation of these technologies
(\citeproc{ref-brynjolfsson_what_2017}{Brynjolfsson and Mitchell, 2017,
p. 1533f}.). Additionally, the adoption of AI technology is found to
significantly alter the skill-demand distribution of firms, with the
number of previously highly demanded skills declining while
simultaneously creating demand for new skills
(\citeproc{ref-acemoglu_ai_2020}{Acemoglu et al., 2020a, p. 19}).

\subsection{Changes of Occupational composition (no net displacement but
need to
re-skill)}\label{changes-of-occupational-composition-no-net-displacement-but-need-to-re-skill}

Furthermore, it is important to note that previous research on the
effects of robots, software and AI - that have been summarized under the
umbrella term ``automation'' (\citeproc{ref-mann_benign_2018}{Mann and
Püttmann, 2018, p. 40}) - in general may not have found net negative
effects on employment but a restructuring of composition of occupations.
The aforementioned study from Autor et al.
(\citeproc{ref-autor_untangling_2015}{2015}) {[}p.~644{]} found
automation, while having no aggregate effects on employment, lead to a
decline in occupations involving routine tasks and and an increase in
non-routine (abstract) tasks. The same effect was found in Graetz and
Michaels (\citeproc{ref-graetz_robots_2015}{2015, p. 766}) studying the
introduction of industrial robots.\\
These effects remain only harmless as long as the assumption holds true
that displaced labor can in fact always reallocate itself to new tasks.
Should this assumption be contradicted, and the the negative effects of
automation on employment are no longer offset by the positive effects of
reallocation, the phenomenon of occupational migration would turn into
an observation of job destruction.

\subsection{Changes in labor share}\label{changes-in-labor-share}

The introduction of capital, whether to complement or substitute labor,
intuitively leads to a decline of a firms profits paid to labor as the
share of labors input relative to the output value decreases. And in
fact Karabarbounis and Neiman
(\citeproc{ref-karabarbounis_global_2014}{2014}) {[}p.~99{]} show that
the observed decline in capital prices explains almost half the decline
in global labor share, that has been observed in recent decades. This
might seem problematic as an increasing portion of a firms revenue
remains as corporate profits and savings (given that the capital
invested leads to a decrease in marginal costs - through substitution of
labor and/ or increased production) rather than being redistributed to
labor. Karabarbounis and Neiman
(\citeproc{ref-karabarbounis_global_2014}{2014, p. 102}) further show
that the observed decline in labor share is accompanied by an increase
in corporate revenue and savings. This is also brought forward from
Acemoglu and Restrepo (\citeproc{ref-acemoglu_automation_2019}{2019})
{[}p.~27{]} who conclude that ``{[}\ldots{]} automation always reduces
the labor share and may reduce labor demand {[}\ldots{]}'' but also
mention that the creation of new tasks necessarily increases the labor
share. These results where further solidified by Acemoglu et al.
(\citeproc{ref-acemoglu_competing_2020}{2020b, p. 387}) who investigated
the French manufacturing market and found firms exposed to automation
(in this study measured by the introduction of robots) to experience
significant declines in their labor share.

\subsection{Summary of the different findings -\textgreater{} why
findings
differ}\label{summary-of-the-different-findings---why-findings-differ}

The net impact assessment of automation on socioeconomic factors widely
differs in the aforementioned literature (see also
\citeproc{ref-frank_toward_2019}{Frank et al., 2019, p. 6532}). Some
research has focused on local labor markets (commuting zones) (see
\citeproc{ref-acemoglu_robots_2020}{Acemoglu and Restrepo, 2020a};
\citeproc{ref-autor_untangling_2015}{Autor et al., 2015};
\citeproc{ref-autor_growth_2013}{Autor and Dorn, 2013}), while other
research has researched national effects {[}see xxx{]} and international
effects (see \citeproc{ref-graetz_robots_2018}{Graetz and Michaels,
2018}). While one would expect to see the same relationship between the
chosen variables on all levels and apart from differences in research
design, it may be difficult to assess effects on a greater aggregate
level as the number of variables that would need to be included to
account for differences between and within groups becomes unfeasible.

\subsection{Definitions of AI}\label{definitions-of-ai}

The classification if Artificial Intelligence remains also difficult due
to the fact that there is yet no widespread agreement on the definition
of intelligence itself (\citeproc{ref-legg_collection_2007}{Legg and
Hutter, 2007}).

\section{Methodology}\label{methodology}

The methodological approach has similarities to Mann and Püttmann
(\citeproc{ref-mann_benign_2018}{2018, p. 13}) who used patent counts as
a proxy for estimating the level of automation present within a US
commuting zone. However, the method of selecting patents differs. While
Mann and Püttmann (\citeproc{ref-mann_benign_2018}{2018}) classified
texts based on the tasks they may effect within occupations, the
presented approach here uses API query composition to preselect patents
whose title or abstract match keywords reserved to an industry.

\subsection{Data Sources}\label{data-sources}

Data about patent publications is obtained from the European Patent
Office's Open Patent Services (OPS) API
(\citeproc{ref-european_patent_office_open_2023}{European Patent Office,
2023}) as well as the Annual Structural Business Statistics (SBS) by
Eurostat (\citeproc{ref-eurostat_annual_2023}{Eurostat, 2023a}).
Furthermore, Eurostats code lists of Statistical classification of
economic activities in the European Community (NACE Revision 2)
(\citeproc{ref-eurostat_statistical_2023}{Eurostat, 2023b}) (henceforth
``NACE'') and Economic Indicators for Eurostat's SBS
(\citeproc{ref-eurostat_economical_2023}{Eurostat, 2023c}). While there
are a variety of possible technologies that may fall under the umbrella
term ``Artificial Intelligence'', as this research aims to assess AI's
socioeconomic impact - which, if negative, falls into the governmental
realm - a legal definition of AI is preferable as a classifier.
Furthermore, it is arguable that the political definition is likely to
have the greatest (socio)economic impact in the near future due to
possible (and probable) regulation. As there is no legal definition yet
- at least in the EU - technologies listed in the European Commisions
latest proposal for the ``Artificial Intelligence Act{[}'s{]}''
(\citeproc{ref-european_commission_proposal_2021}{European Commission,
2021}) annex (\citeproc{ref-european_commission_annexes_202}{European
Commission, 202AD}) will be used.\footnote{While this proposal is not
  yet in effect, it is likely to be adopted in the near future and is
  therefore used as a proxy for a legal definition of AI.}

Additionally Cooperative Patent Classification (CPC) codes are used to
retrieve patents that utilize artificial intelligence technology. As
there is no clear mapping between the European Commission's definition
and CPC codes, classifications are chosen to the author's best
knowledge.

\phantomsection\label{tbl-cpc-codes}
\begin{longtable}[]{@{}ll@{}}
\caption{\label{tbl-cpc-codes}Selected CPC Codes}\tabularnewline
\toprule\noalign{}
Class & CPC \\
\midrule\noalign{}
\endfirsthead
\toprule\noalign{}
Class & CPC \\
\midrule\noalign{}
\endhead
\bottomrule\noalign{}
\endlastfoot
Machine Learning & G06N20/00, G06N20/10, G06N20/20 \\
Supervised Learning & G06N3/09 \\
Unsupervised Learning & G06N3/088 \\
Reinforcement Learning & G06N3/092 \\
Deep Learning & G06N3/08 \\
\end{longtable}

\subsection{Data Acquisition}\label{data-acquisition}

In order to retrieve data from the European Patent Office's Open Patent
Services (OPS) API, queries were composed to link retrieved patents to
their respective industry. The query composition is based on the
selected CPC codes displayed in Table~\ref{tbl-cpc-codes} as well as
keywords from the list of NACE codes that have been retrieved from
Eurostat. Each NACE code is composed of section (alphabetical), division
(numerical), group (numerical) and class (numerical) of a particular
economic activity. Sections relate to the overall indsutry, while
divisions, groups and classes relate to more specific activities within
the industry (\citeproc{ref-nacebac}{\textbf{nacebac?}}). For each
industry, keywords are extracted from the NACE code's description. To
ensure only relevant keywords are used, each description is cleaned of
common characters and unrelated words (e.g., ``,'', ``and'', ``or'',
``to''). Descriptions for each industry are then split into lists of
single keywords that will be used in the API query.

Because some industries contain a variety of different activities (e.g.,
NACE industry (section) ``A'' relates to ``Agriculture, forestry and
fishing'' (\citeproc{ref-eurostat_economical_2023}{Eurostat, 2023c})),
main keywords that relate to the section as a whole are manually
selected (see Table~\ref{tbl-nacemainkeywords} in the
\nameref{sec-appendix}). For each industry and main keyword, queries are
then build using the (manuallly selected) main keyword, the description
keywords, as well as the chosen CPC codes. The resulting query is then
used to retrieve patents from the OPS API. To limit query results to the
relevant market, only patents filed with the European Patent Office (EP)
will be retrieved. This approach disregards patents filed with national
patent offices. However, since the query language is english, and many
national patents are only filed in their native language, retrieving
patents for all EU markets was unfeasible. The query is composed of the
following elements:

\begin{quote}
\textbf{(ta = Main Keyword) AND (ta = ANY Description Keywords) AND (cpc
ANY CPC Codes) AND (ap = ``EP'')}\\
\emph{Note: ta = title or abstract; ap = Application Number, referring
to the Patent Office the patent was filed at. In this case, ``EP''
refers to the European Patent Office. See Table~\ref{tbl-queryexample}
for example queries}
\end{quote}

The queries are then posted to the OPS API's Published Data Keywords
Search with Variable Constituents endpoint
(\citeproc{ref-european_patent_office_published_nodate}{European Patent
Office, n.d.}) and data from the responses - which are provided in JSON
format - extracted.

\subsection{Hypothesis}\label{hypothesis}

\begin{enumerate}
\def\labelenumi{\arabic{enumi}.}
\tightlist
\item
  Patents do not affect wage adjusted labor productivity
\item
  Patents do not affect share of personnel costs in production
  (percentage)
\item
  Patents do not affect number of people employed
\item
  Patents do not affect gross value added per employee
\end{enumerate}

In order to determine the effects of

\section{Results}\label{results}

\section{Discussion}\label{discussion}

\section{Limitations}\label{limitations}

As pointed out by Trajtenberg
(\citeproc{ref-trajtenberg_penny_1990}{1990}), the plain number of
patent counts disregard the fact that patents to not carry equal
economical weight, i.e., the effect a patent might have on a market or
industry cannot be inferred by the presence of a patent without
incorporating weights.

\section{Conclusion}\label{conclusion}

\newpage{}

\section*{References}\label{sec-references}
\addcontentsline{toc}{section}{References}

\phantomsection\label{refs}
\begin{CSLReferences}{1}{0}
\bibitem[\citeproctext]{ref-acemoglu_harms_2021}
Acemoglu, D., 2021. Harms of {AI}. Department of Economics,
Massachusetts Institute of Technology.

\bibitem[\citeproctext]{ref-acemoglu_chapter_2011}
Acemoglu, D., Autor, D., 2011. Chapter 12 - {Skills}, {Tasks} and
{Technologies}: {Implications} for {Employment} and {Earnings}**{We}
thank {Amir} {Kermani} for outstanding research assistance and {Melanie}
{Wasserman} for persistent, meticulous and ingenious work on all aspects
of the chapter. {We} are indebted to {Arnaud} {Costinot} for insightful
comments and suggestions. {Autor} acknowledges support from the
{National} {Science} {Foundation} ({CAREER} award {SES}-0239538)., in:
Card, D., Ashenfelter, O. (Eds.), Handbook of {Labor} {Economics}.
Elsevier, pp. 1043--1171.
\url{https://doi.org/10.1016/S0169-7218(11)02410-5}

\bibitem[\citeproctext]{ref-acemoglu_ai_2020}
Acemoglu, D., Autor, D., Hazell, J., Restrepo, P., 2020a. {AI} and
{Jobs}: {Evidence} from {Online} {Vacancies}. Working {Paper} {Series}.
\url{https://doi.org/10.3386/w28257}

\bibitem[\citeproctext]{ref-acemoglu_competing_2020}
Acemoglu, D., Lelarge, C., Restrepo, P., 2020b. Competing with {Robots}:
{Firm}-{Level} {Evidence} from {France}. AEA Papers \& Proceedings 110,
383--388. \url{https://doi.org/10.1257/pandp.20201003}

\bibitem[\citeproctext]{ref-acemoglu_tasks_2022}
Acemoglu, D., Restrepo, P., 2022. Tasks, {Automation}, and the {Rise} in
{U}.{S}. {Wage} {Inequality}. Econometrica 90, 1973--2016.
\url{https://doi.org/10.3982/ECTA19815}

\bibitem[\citeproctext]{ref-acemoglu_robots_2020}
Acemoglu, D., Restrepo, P., 2020a. Robots and {Jobs}: {Evidence} from
{US} {Labor} {Markets}. Journal of Political Economy 128, 2188--2244.
\url{https://doi.org/10.1086/705716}

\bibitem[\citeproctext]{ref-acemoglu_unpacking_2020}
Acemoglu, D., Restrepo, P., 2020b. Unpacking {Skill} {Bias}:
{Automation} and {New} {Tasks}. AEA Papers \& Proceedings 110, 356--361.
\url{https://doi.org/10.1257/pandp.20201063}

\bibitem[\citeproctext]{ref-acemoglu_automation_2019}
Acemoglu, D., Restrepo, P., 2019. Automation and {New} {Tasks}: {How}
{Technology} {Displaces} and {Reinstates} {Labor}. Journal of Economic
Perspectives 33, 3--30. \url{https://doi.org/10.1257/jep.33.2.3}

\bibitem[\citeproctext]{ref-acemoglu_race_2018}
Acemoglu, D., Restrepo, P., 2018a. The {Race} between {Man} and
{Machine}: {Implications} of {Technology} for {Growth}, {Factor}
{Shares}, and {Employment}. American Economic Review 108, 1488--1542.
\url{https://doi.org/10.1257/aer.20160696}

\bibitem[\citeproctext]{ref-acemoglu_artificial_2018}
Acemoglu, D., Restrepo, P., 2018b.
\href{https://www.nber.org/books-and-chapters/economics-artificial-intelligence-agenda/artificial-intelligence-automation-and-work}{Artificial
{Intelligence}, {Automation}, and {Work}}, in: The {Economics} of
{Artificial} {Intelligence}: {An} {Agenda}. University of Chicago Press,
pp. 197--236.

\bibitem[\citeproctext]{ref-acemoglu_low-skill_2017}
Acemoglu, D., Restrepo, P., 2017. Low-{Skill} and {High}-{Skill}
{Automation}. \url{https://doi.org/10.2139/ssrn.3083552}

\bibitem[\citeproctext]{ref-anderson_automation_2017}
Anderson, A.S. and M., 2017.
\href{https://www.pewresearch.org/internet/2017/10/04/automation-in-everyday-life/}{Automation
in {Everyday} {Life}}. Pew Research Center: Internet, Science \& Tech.

\bibitem[\citeproctext]{ref-arntz_risk_2016}
Arntz, M., Gregory, T., Zierahn, U., 2016. The {Risk} of {Automation}
for {Jobs} in {OECD} {Countries}: {A} {Comparative} {Analysis}. OECD,
Paris. \url{https://doi.org/10.1787/5jlz9h56dvq7-en}

\bibitem[\citeproctext]{ref-autor_growth_2013}
Autor, D.H., Dorn, D., 2013. The {Growth} of {Low}-{Skill} {Service}
{Jobs} and the {Polarization} of the {US} {Labor} {Market}. American
Economic Review 103, 1553--1597.
\url{https://doi.org/10.1257/aer.103.5.1553}

\bibitem[\citeproctext]{ref-autor_untangling_2015}
Autor, D.H., Dorn, D., Hanson, G.H., 2015. Untangling {Trade} and
{Technology}: {Evidence} from {Local} {Labour} {Markets}. Economic
Journal 125, 621--646. \url{https://doi.org/10.1111/ecoj.12245}

\bibitem[\citeproctext]{ref-beaudry_great_2016}
Beaudry, P., Green, D.A., Sand, B.M., 2016. The {Great} {Reversal} in
the {Demand} for {Skill} and {Cognitive} {Tasks}. Journal of Labor
Economics 34, S199--S247. \url{https://doi.org/10.1086/682347}

\bibitem[\citeproctext]{ref-boustan_automation_2022}
Boustan, L.P., Choi, J., Clingingsmith, D., 2022. Automation {After} the
{Assembly} {Line}: {Computerized} {Machine} {Tools}, {Employment} and
{Productivity} in the {United} {States}. Working {Paper} {Series}.
\url{https://doi.org/10.3386/w30400}

\bibitem[\citeproctext]{ref-brynjolfsson_what_2017}
Brynjolfsson, E., Mitchell, T., 2017. What can machine learning do?
{Workforce} implications. Science 358, 1530--1534.
\url{https://doi.org/10.1126/science.aap8062}

\bibitem[\citeproctext]{ref-brynjolfsson_what_2018}
Brynjolfsson, E., Mitchell, T., Rock, D., 2018. What {Can} {Machines}
{Learn} and {What} {Does} {It} {Mean} for {Occupations} and the
{Economy}? AEA Papers \& Proceedings 108, 43--47.
\url{https://doi.org/10.1257/pandp.20181019}

\bibitem[\citeproctext]{ref-cirera_effects_2019}
Cirera, X., Sabetti, L., 2019. The effects of innovation on employment
in developing countries: Evidence from enterprise surveys. Industrial
and Corporate Change 28, 161--176.
\url{https://doi.org/10.1093/icc/dty061}

\bibitem[\citeproctext]{ref-dauth_german_2017}
Dauth, W., Findeisen, S., Südekum, J., Woessner, N., 2017.
\href{https://ssrn.com/abstract=3039031}{German {Robots} - {The}
{Impact} of {Industrial} {Robots} on {Workers}}. IAB Discussion Paper
30/2017.

\bibitem[\citeproctext]{ref-dauth_adjustment_2021}
Dauth, W., Findeisen, S., Suedekum, J., Woessner, N., 2021. The
{Adjustment} of {Labor} {Markets} to {Robots}. Journal of the European
Economic Association 19, 3104--3153.
\url{https://doi.org/10.1093/jeea/jvab012}

\bibitem[\citeproctext]{ref-european_commission_proposal_2021}
European Commission, 2021.
\href{https://eur-lex.europa.eu/legal-content/EN/TXT/HTML/?uri=CELEX:52021PC0206}{Proposal
for a {REGULATION} {OF} {THE} {EUROPEAN} {PARLIAMENT} {AND} {OF} {THE}
{COUNCIL}}.

\bibitem[\citeproctext]{ref-european_commission_annexes_202}
European Commission, 202AD. {ANNEXES} to the {Proposal} for a
{Regulation} of the {European} {Parliament} and of the {Council}.

\bibitem[\citeproctext]{ref-european_patent_office_published_nodate}
European Patent Office, n.d.
\href{https://developers.epo.org/ops-v3-2/apis/get/published-data/search/\%7Bconstituent\%7D}{Published
{Data} {Keywords} {Search} with {Variable} {Constituents} {\textbar}
{EPO} {Developer} {Portal}}.

\bibitem[\citeproctext]{ref-european_patent_office_open_2023}
European Patent Office, 2023.
\href{https://developers.epo.org/ops-v3-2/apis}{Open {Patent} {Services}
({OPS})}.

\bibitem[\citeproctext]{ref-eurostat_annual_2023}
Eurostat, 2023a.
\href{https://ec.europa.eu/eurostat/databrowser/view/sbs_na_sca_r2/default/table?lang=en}{Annual
enterprise statistics for special aggregates of activities ({NACE}
{Rev}. 2)}.

\bibitem[\citeproctext]{ref-eurostat_statistical_2023}
Eurostat, 2023b.
\href{https://ec.europa.eu/eurostat/api/dissemination/sdmx/2.1/codelist/ESTAT/NACE_R2/?compressed=true&format=TSV&lang=en}{Statistical
classification of economic activities in the {European} {Community}
({NACE} {Rev}. 2)}.

\bibitem[\citeproctext]{ref-eurostat_economical_2023}
Eurostat, 2023c.
\href{https://ec.europa.eu/eurostat/api/dissemination/sdmx/2.1/codelist/ESTAT/INDIC_SB/?compressed=true&format=TSV&lang=en}{Economical
indicator for structural business statistics}.

\bibitem[\citeproctext]{ref-felten_effect_2019}
Felten, E.W., Raj, M., Seamans, R., 2019. The {Effect} of {Artificial}
{Intelligence} on {Human} {Labor}: {An} {Abilitybased} {Approach}.
Academy of Management Annual Meeting Proceedings 2019, 791--796.
\url{https://doi.org/10.5465/AMBPP.2019.140}

\bibitem[\citeproctext]{ref-frank_toward_2019}
Frank, M.R., Autor, D., Bessen, J.E., Brynjolfsson, E., Cebrian, M.,
Deming, D.J., Feldman, M., Groh, M., Lobo, J., Moro, E., Wang, D., Youn,
H., Rahwan, I., 2019. Toward understanding the impact of artificial
intelligence on labor. Proceedings of the National Academy of Sciences
116, 6531--6539. \url{https://doi.org/10.1073/pnas.1900949116}

\bibitem[\citeproctext]{ref-frey_future_2017}
Frey, C.B., Osborne, M.A., 2017. The future of employment: {How}
susceptible are jobs to computerisation? Technological Forecasting and
Social Change 114, 254--280.
\url{https://doi.org/10.1016/j.techfore.2016.08.019}

\bibitem[\citeproctext]{ref-graetz_robots_2018}
Graetz, G., Michaels, G., 2018. Robots at {Work}. Review of Economics \&
Statistics 100, 753--768. \url{https://doi.org/10.1162/rest_a_00754}

\bibitem[\citeproctext]{ref-graetz_robots_2015}
Graetz, G., Michaels, G., 2015.
\href{https://ideas.repec.org//p/cep/cepcnp/447.html}{Robots at work:
The impact on productivity and jobs}. CentrePiece - The magazine for
economic performance.

\bibitem[\citeproctext]{ref-karabarbounis_global_2014}
Karabarbounis, L., Neiman, B., 2014. The {Global} {Decline} of the
{Labor} {Share}. The Quarterly Journal of Economics 129, 61--103.
\url{https://doi.org/10.1093/qje/qjt032}

\bibitem[\citeproctext]{ref-legg_collection_2007}
Legg, S., Hutter, M., 2007. \href{http://arxiv.org/abs/0706.3639}{A
{Collection} of {Definitions} of {Intelligence}}.

\bibitem[\citeproctext]{ref-mann_benign_2018}
Mann, K., Püttmann, L., 2018.
\href{https://ssrn.com/abstract=2959584}{Benign {Effects} of
{Automation}: {New} {Evidence} {From} {Patent} {Texts}}.

\bibitem[\citeproctext]{ref-mokyr_history_2015}
Mokyr, J., Vickers, C., Ziebarth, N.L., 2015. The {History} of
{Technological} {Anxiety} and the {Future} of {Economic} {Growth}: {Is}
{This} {Time} {Different}? Journal of Economic Perspectives 29, 31--50.
\url{https://doi.org/10.1257/jep.29.3.31}

\bibitem[\citeproctext]{ref-pajarinen_computerization_2015}
Pajarinen, M., Rouvinen, P., Ekeland, A., 2015.
\href{https://ideas.repec.org//p/rif/briefs/34.html}{Computerization
{Threatens} {One}-{Third} of {Finnish} and {Norwegian} {Employment}}.
ETLA Brief.

\bibitem[\citeproctext]{ref-trajtenberg_penny_1990}
Trajtenberg, M., 1990. A {Penny} for {Your} {Quotes}: {Patent}
{Citations} and the {Value} of {Innovations}. The RAND Journal of
Economics 21, 172. \url{https://doi.org/10.2307/2555502}

\bibitem[\citeproctext]{ref-webb_impact_2019}
Webb, M., 2019. The {Impact} of {Artificial} {Intelligence} on the
{Labor} {Market}. \url{https://doi.org/10.2139/ssrn.3482150}

\end{CSLReferences}

\newpage{} \# Appendix \{\#sec-appendix .unnumbered\}

\phantomsection\label{tbl-nacemainkeywords}
\begin{longtable}[]{@{}ll@{}}
\caption{\label{tbl-nacemainkeywords}Selected main keywords for NACE
industries}\tabularnewline
\toprule\noalign{}
NACE Section & Keywords \\
\midrule\noalign{}
\endfirsthead
\toprule\noalign{}
NACE Section & Keywords \\
\midrule\noalign{}
\endhead
\bottomrule\noalign{}
\endlastfoot
A & AGRICULTURE, FORESTRY, FISHING \\
B & MINING, QUARRYING \\
C & MANUFACTURING \\
D & ELECTRICITY, GAS, STEAM, AIR CONDITIONING SUPPLY \\
E & WATER SUPPLY, SEWERAGE, WASTE MANAGEMENT, \\
& REMEDIATION ACTIVITIES \\
F & CONSTRUCTION \\
G & WHOLESALE, RETAIL, REPAIR OF MOTOR VEHICLES AND \\
& MOTORCYCLES \\
H & TRANSPORTATION, STORAGE \\
I & ACCOMMODATION, FOOD SERVICE ACTIVITIES \\
J & INFORMATION, COMMUNICATION \\
K & FINANCIAL ACTIVITIES, INSURANCE ACTIVITIES \\
L & REAL ESTATE \\
M & PROFESSIONAL ACTIVITIES, SCIENTIFIC ACTIVITIES, \\
& TECHNICAL ACTIVITIES \\
N & ADMINISTRATIVE ACTIVITIES, SUPPORT SERVICE \\
& ACTIVITIES \\
O & PUBLIC ADMINISTRATION, DEFENCE, COMPULSORY SOCIAL \\
& SECURITY \\
P & EDUCATION \\
Q & HUMAN HEALTH, SOCIAL WORK ACTIVITIES \\
R & ARTS, ENTERTAINMENT, RECREATION \\
S & OTHER SERVICE ACTIVITIES \\
T & ACTIVITIES OF HOUSEHOLDS AS EMPLOYERS, \\
& UNDIFFERENTIATED GOODS, SERVICES PRODUCING \\
& ACTIVITIES OF HOUSEHOLDS FOR OWN USE \\
U & ACTIVITIES OF EXTRATERRITORIAL ORGANISATIONS, \\
& ACTIVITIES OF EXTRATERRITORIAL BODIES \\
\end{longtable}

\phantomsection\label{tbl-queryexample}
\begin{longtable}[]{@{}
  >{\raggedleft\arraybackslash}p{(\columnwidth - 2\tabcolsep) * \real{0.0131}}
  >{\raggedright\arraybackslash}p{(\columnwidth - 2\tabcolsep) * \real{0.9869}}@{}}
\caption{\label{tbl-queryexample}Example queries posted to the OPS
API}\tabularnewline
\toprule\noalign{}
\begin{minipage}[b]{\linewidth}\raggedleft
\end{minipage} & \begin{minipage}[b]{\linewidth}\raggedright
Query examples
\end{minipage} \\
\midrule\noalign{}
\endfirsthead
\toprule\noalign{}
\begin{minipage}[b]{\linewidth}\raggedleft
\end{minipage} & \begin{minipage}[b]{\linewidth}\raggedright
Query examples
\end{minipage} \\
\midrule\noalign{}
\endhead
\bottomrule\noalign{}
\endlastfoot
0 & (ta ALL ``agriculture'' OR ta ALL ``forestry'' OR ta ALL
``fishing'') AND (ta = ``farming'' OR ta = ``raising'' OR ta =
``related'' OR ta = ``dairy'' OR ta = ``grapes'' OR ta = ``oleaginous''
OR ta = ``buffaloes'' ) AND cpc any ``G06N20/00 G06N20/10 G06N20/20
G06N3/09 G06N3/088 G06N3/092 G06N3/08'' AND AP=``EP'' \\
1 & (ta ALL ``agriculture'' OR ta ALL ``forestry'' OR ta ALL
``fishing'') AND (ta = ``leguminous'' OR ta = ``beverage'' OR ta =
``pharmaceutical'' OR ta = ``hunting'' OR ta = ``tree'' OR ta =
``sheep'' OR ta = ``pome'' ) AND cpc any ``G06N20/00 G06N20/10 G06N20/20
G06N3/09 G06N3/088 G06N3/092 G06N3/08'' AND AP=``EP'' \\
2 & (ta ALL ``agriculture'' OR ta ALL ``forestry'' OR ta ALL
``fishing'') AND (ta = ``propagation'' OR ta = ``camelids'' OR ta =
``crop'' OR ta = ``stone'' OR ta = ``tropical'' OR ta = ``oil'' OR ta =
``service'' ) AND cpc any ``G06N20/00 G06N20/10 G06N20/20 G06N3/09
G06N3/088 G06N3/092 G06N3/08'' AND AP=``EP'' \\
3 & (ta ALL ``agriculture'' OR ta ALL ``forestry'' OR ta ALL
``fishing'') AND (ta = ``roots'' OR ta = ``bush'' OR ta =
``post-harvest'' OR ta = ``tobacco'' OR ta = ``cereals'' OR ta =
``cattle'' OR ta = ``crops'' ) AND cpc any ``G06N20/00 G06N20/10
G06N20/20 G06N3/09 G06N3/088 G06N3/092 G06N3/08'' AND AP=``EP'' \\
4 & (ta ALL ``agriculture'' OR ta ALL ``forestry'' OR ta ALL
``fishing'') AND (ta = ``spices'' OR ta = ``perennial'' OR ta =
``subtropical'' OR ta = ``except'' OR ta = ``camels'' OR ta =
``support'' OR ta = ``plant'' ) AND cpc any ``G06N20/00 G06N20/10
G06N20/20 G06N3/09 G06N3/088 G06N3/092 G06N3/08'' AND AP=``EP'' \\
5 & (ta ALL ``agriculture'' OR ta ALL ``forestry'' OR ta ALL
``fishing'') AND (ta = ``melons'' OR ta = ``poultry'' OR ta =
``animals'' OR ta = ``vegetables'' OR ta = ``nuts'' OR ta =
``activities'' OR ta = ``trapping'' ) AND cpc any ``G06N20/00 G06N20/10
G06N20/20 G06N3/09 G06N3/088 G06N3/092 G06N3/08'' AND AP=``EP'' \\
6 & (ta ALL ``agriculture'' OR ta ALL ``forestry'' OR ta ALL
``fishing'') AND (ta = ``agriculture'' OR ta = ``production'' OR ta =
``processing'' OR ta = ``citrus'' OR ta = ``aromatic'' OR ta = ``rice''
OR ta = ``non-perennial'' ) AND cpc any ``G06N20/00 G06N20/10 G06N20/20
G06N3/09 G06N3/088 G06N3/092 G06N3/08'' AND AP=``EP'' \\
7 & (ta ALL ``agriculture'' OR ta ALL ``forestry'' OR ta ALL
``fishing'') AND (ta = ``horses'' OR ta = ``goats'' OR ta = ``fruits''
OR ta = ``seeds'' OR ta = ``equines'' OR ta = ``seed'' OR ta = ``drug''
) AND cpc any ``G06N20/00 G06N20/10 G06N20/20 G06N3/09 G06N3/088
G06N3/092 G06N3/08'' AND AP=``EP'' \\
8 & (ta ALL ``agriculture'' OR ta ALL ``forestry'' OR ta ALL
``fishing'') AND (ta = ``fibre'' OR ta = ``growing'' OR ta = ``sugar''
OR ta = ``mixed'' OR ta = ``cane'' OR ta = ``tubers'' OR ta = ``pigs'' )
AND cpc any ``G06N20/00 G06N20/10 G06N20/20 G06N3/09 G06N3/088 G06N3/092
G06N3/08'' AND AP=``EP'' \\
9 & (ta ALL ``agriculture'' OR ta ALL ``forestry'' OR ta ALL
``fishing'') AND (ta = ``swine'' OR ta = ``animal'' ) AND cpc any
``G06N20/00 G06N20/10 G06N20/20 G06N3/09 G06N3/088 G06N3/092 G06N3/08''
AND AP=``EP'' \\
\end{longtable}



\end{document}
